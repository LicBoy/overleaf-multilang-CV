%% ----- HEADER SECTION -----
\newcommand{\name}
{\langen{Ruslan Akhmetvaleev}
\langru{Руслан Ахметвалеев}}

\newcommand{\underNameStatus}{\langen{Looking for a nice job}\langru{В поиске класной работы}}

\newcommand{\mobileWord}{\langen{Mobile}\langru{Тел. номер}}
\newcommand{\locationWord}{\langen{Location}\langru{Локация}}
\newcommand{\location}{\langen{Almaty}\langru{Алматы}}


%% ----- EDUCATION SECTION -----
\newcommand{\educationWord}{\langen{Education}\langru{Образование}}

\newcommand{\uniDegreeBachelor}{\langen{Bachelor}\langru{Бакалавриат}}

\newcommand{\uniFirstName}
{\langen{Innopolis University}\langru{Университет Иннополис}}
\newcommand{\uniFirstLocation}
{\langen{Innopolis, Tatarstan}\langru{Иннополис, Татарстан}}
\newcommand{\uniFirstDegree}
{\langen{\uniDegreeBachelor, Computer Science}\langru{\uniDegreeBachelor, Информатика и вычислительная техника}}
\newcommand{\uniFirstDate}
{\langen{Aug. 2017 -- June 2018}\langru{Aвг. 2017 -- Июнь 2018}}
\newcommand{\uniSecondName}
{\langen{Kazan State Power Engineering University}\langru{Казанский Государственный Энергетический Университет}}
\newcommand{\uniSecondLocation}
{\langen{Kazan, Tatarstan}\langru{Казань, Татарстан}}
\newcommand{\uniSecondDegree}
{\langen{\uniDegreeBachelor, Applied Mathematics}\langru{\uniDegreeBachelor, Прикладная математика}}
\newcommand{\uniSecondDate}
{\langen{Sep. 2018 -- July 2022}\langru{Сен. 2018 -- Июль 2022}}


%% ----- WORK EXPERIENCE SECTION -----
\newcommand{\expWord}{\langen{Work experience}\langru{Опыт работы}}

\newcommand{\jobThirdName}
{\langen{CMA Small Systems AB}\langru{CMA Small Systems AB}}
\newcommand{\jobThirdPosition}
{\langen{QA Automation Engineer}\langru{QA Automation Engineer}}
\newcommand{\jobThirdLocation}
{\langen{Remote}\langru{Удалённо}}
\newcommand{\jobThirdDate}
{\langen{June 2023 - Present}\langru{Июнь 2023 - наст. время}}
\newcommand{\jobThirdItemFirstTitle}
{\langen{Framework and stack}\langru{Фреймворк и стэк}}
\newcommand{\jobThirdItemFirstDesc}
{\langen{Development of a framework for testing the backend: SQL queries to the database, web queries to the company’s services. Writing autotests for the backend in Python+pytest, 'cx\textunderscore Oracle' module for working with Oracle DB, 'requests' module for web requests. Development of our own tool for parallel test launch using only built-in Python modules, which significantly reduced the time to fully run autotests.}\langru{Разработка фреймворка для тестирования бэкенда: SQL запросы к БД, веб-запросы к сервисам компании. Написание автотестов для бэкенда на Python+pytest, модуль 'cx\textunderscore Oracle' для работы с Oracle DB, модуль 'requests' для веб-запросов. Разработка собственного инструмента для параллельного запуска тестов с использованием только встроенных модулей Python, что значительно уменьшило время полного запуска автотестов.}}
\newcommand{\jobThirdItemSecondTitle}
{\langen{Fintech testing}\langru{Финтех тестирование}}
\newcommand{\jobThirdItemSecondDesc}
{\langen{Testing and sending transactions related to banking operations. Working with SWIFT, MX and MT formats, preparing documents for various interbank transactions and certificates for message validation.}\langru{Тестирование и отправка транзакций, связанных с банковскими операциями. Работа со SWIFT, форматами MX и MT, подготовка документов для разных межбанковских операций и сертификатов для валидации сообщений.}}
\newcommand{\jobThirdItemThirdTitle}
{\langen{Jenkins and reporting}\langru{Jenkins и отчёты}}
\newcommand{\jobThirdItemThirdDesc}
{\langen{Deployment of autotests on Jenkins using a pipeline, parameterization in the Jenkinsfile for a more customized launch of autotests, for example in multi-threaded mode, for different groups of tests, etc. And generation of detailed Allure reports for the project team, so that the cause of each bug is found as quickly as possible.}\langru{Развёртывание автотестов на Jenkins с помощью пайплайна, параметризация в Jenkins-файле для более кастомизированного запуска автотестов, например в многопоточном режиме, для разных групп тестов и т.д. Также формирование детализированных Allure-отчётов для команды проекта, так что причина каждого бага находится в быстрейшие сроки.}}

\newcommand{\jobFirstName}
{\langen{A1QA}\langru{A1QA}}
\newcommand{\jobFirstPosition}
{\langen{Automation Testing Engineer}\langru{Инженер по автоматизации тестирования}}
\newcommand{\jobFirstLocation}
{\langen{Remote}\langru{Удалённо}}
\newcommand{\jobFirstDate}
{\langen{Jan. 2023 - June 2023}\langru{Янв. 2023 - Июнь 2023}}
\newcommand{\jobFirstItemFirstTitle}
{\langen{Framework}\langru{Фреймворк}}
\newcommand{\jobFirstItemFirstDesc}
{\langen{Writing autotests for test scenarios using Node.js (mocha + chai), Selenium and WebdriverIO. Develop and maintain your own testing framework with multiple templates: PageObject, BaseElement, BrowserFactory, etc.}\langru{Написание автотестов по тестовым сценариям с использованием Node.js (mocha+chai), Selenium и WebdriverIO. Разработка и поддержка своего фреймворка тестирования с несколькими шаблонами: PageObject, BaseElement, BrowserFactory и т. д.}}
\newcommand{\jobFirstItemSecondTitle}
{\langen{UI Testing}\langru{Тестирование UI}}
\newcommand{\jobFirstItemSecondDesc}
{\langen{Testing the UI functionality of websites, REST API functionality. Working with the developer console, selecting the necessary selectors by Xpath, CSS locators.}\langru{Тестирование функциональности UI веб-сайтов, функциональности REST API. Работа с консолью разработчика, выбор нужных селекторов через Xpath, CSS локаторы.}}
\newcommand{\jobFirstItemThirdTitle}
{\langen{CI, SQL, Docker}\langru{CI, SQL, Docker}}
\newcommand{\jobFirstItemThirdDesc}
{\langen{Testing using various APIs (eg Google API), testing using MySQL databases. I worked with Docker, I know the basic linux commands, I can deploy a basic application. Worked with Jenkins CI: build settings, build triggers (schedule, change in VCS, another job), testing reports, such as HTML reports.}\langru{Тестирование с использованием различных API (например, Google API), тестирование с использованием баз данных MySQL. Работал с Docker, знаю основные команды linux, могу развернуть базовое приложение. Работал с Jenkins CI: настройки билда, триггеры билда (расписание, изменение в VCS, другая джоба), отчёты тестирования, например HTML-отчёты.}}

\newcommand{\jobSecondName}
{\langen{"Revizor Vision" LLC}\langru{ООО "Ревизор Вижен"}}
\newcommand{\jobSecondPosition}
{\langen{Computer Vision Engineer}\langru{Разработчик Компьютерного Зрения}}
\newcommand{\jobSecondLocation}
{\langen{Remote}\langru{Удалённо}}
\newcommand{\jobSecondDate}
{\langen{Oct. 2022 - Dec. 2022}\langru{Окт. 2022 - Дек. 2022}}
\newcommand{\jobSecondItemFirstTitle}
{\langen{Dataset}\langru{Датасет}}
\newcommand{\jobSecondItemFirstDesc}
{\langen{Collection of dataset from prepared cameras. Its processing and markup with OpenCV, Numpy and Pillow.}\langru{Сбор датасета с подготовленных камер. Его обработка и разметка при помощи OpenCV, Numpy и Pillow.}}
\newcommand{\jobSecondItemSecondTitle}
{\langen{Detection Model}\langru{Модель детекции}}
\newcommand{\jobSecondItemSecondDesc}
{\langen{Checking the quality of the detection of the company model. Changing the architecture of the model using PyTorch.}\langru{Проверка качества детекции модели компании. Изменение архитектуры модели при помощи PyTorch.}}


%% ----- PROJECTS SECTION -----
\newcommand{\projectsWord}{\langen{Projects}\langru{Проекты}}

\newcommand{\projectPositionCreator}
{\langen{Creator}\langru{Создатель}}

\newcommand{\projectFirstName}
{\langen{Super-Resolution (CV) Diploma}\langru{Диплом по задаче Супер-разрешения (CV)}}
\newcommand{\projectFirstDate}
{\langen{Oct. 2021 - June 2022}\langru{Окт. 2021 - Июнь 2022}}
\newcommand{\projectFirstTitleFirst}
{\langen{Theory}\langru{Теория}}
\newcommand{\projectFirstDescFirst}
{\langen{Research of methods for image upscaling, both classical and using neural
networks}\langru{Исследование методов апскейлинга изображений, как классических, так и с использованием нейронных сетей}}
\newcommand{\projectFirstTitleSecond}
{\langen{PyTorch \& Tensorflow}\langru{PyTorch \& Tensorflow}}
\newcommand{\projectFirstDescSecond}
{\langen{Used PyTorch and Tensorflow in process of implementing and estimating neural networks, tried different model architectures, loss functions, optimizators, etc.}\langru{Использовал PyTorch и Tensorflow в процессе реализации и оценки нейронных сетей, пробовал разные архитектуры моделей, функции потерь, оптимизаторы и т.д.}}
\newcommand{\projectFirstTitleThird}
{\langen{Results}\langru{Итоги}}
\newcommand{\projectFirstDescThird}
{\langen{Chose (E)SRGAN as best network, trained them using different architectures, hyperparameters (for instance, for weak PC), applied them for real images, checked them using different metrics.}\langru{Выбрал (E)SRGAN как лучшую сеть, обучил их на разных архитектурах, гиперпараметрах (например, для слабого ПК), применил к реальным изображениям, проверил по разным метрикам.}}

\newcommand{\projectSecondName}
{\langen{Crypto Trading P2P Bot}\langru{Бот для P2P крипто-трейдинга}}
\newcommand{\projectSecondDate}
{\langen{Feb. 2019 - Aug. 2021}\langru{Фев. 2019 - Авг. 2021}}
\newcommand{\projectSecondTitleFirst}
{\langen{Python}\langru{Python}}
\newcommand{\projectSecondDescFirst}
{\langen{Used Python to track bitcoin price on P2P Bitcoin exchange website and perform automated buy/sell operations.}\langru{Использовал Python для отслеживания цены биткойнов на веб-сайте P2P-биржи биткойнов и выполнения автоматических операций покупки/продажи.}}
\newcommand{\projectSecondTitleSecond}
{\langen{Rest API}\langru{Rest API}}
\newcommand{\projectSecondDescSecond}
{\langen{Worked with website's API: web-requests, web-authorization. Implemented specific class for working with API using OOP.}\langru{Работал с API сайта: веб-запросы, веб-авторизация. Реализован специальный класс для работы с API с использованием ООП.}}
\newcommand{\projectSecondTitleThird}
{\langen{Telegram Bot}\langru{Telegram Bot}}
\newcommand{\projectSecondDescThird}
{\langen{Used Telegram API for creating Telegram Bot as UI.}\langru{Использован Telegram API для создания Telegram-бота в качестве пользовательского интерфейса.}}

\newcommand{\projectThirdName}
{\langen{Android Mobile Game}\langru{Мобильная игра для Android}}
\newcommand{\projectThirdDate}
{\langen{Nov. 2020 - Mar. 2021}\langru{Ноя. 2020 - Март 2021}}
\newcommand{\projectThirdTitleFirst}
{\langen{Unity \& C\#}\langru{Unity \& C\#}}
\newcommand{\projectThirdDescFirst}
{\langen{Created mobile game 'Swiper' using Unity3D and C\# with arcade gameplay}\langru{Создал мобильную игру 'Swiper' с использованием Unity3D и C\# с аркадным геймплеем.}}
\newcommand{\projectThirdTitleSecond}
{\langen{Design}\langru{Дизайн}}
\newcommand{\projectThirdDescSecond}
{\langen{Created sprites using Photoshop, self-made design and animations}\langru{Создал спрайты с помощью Photoshop, с нуля созданные дизайн и анимации}}
\newcommand{\projectThirdTitleThird}
{\langen{Advertising}\langru{Реклама}}
\newcommand{\projectThirdDescThird}
{\langen{Integrated interstitial advertising using AdMob service}\langru{Интегрированная межстраничная реклама с помощью сервиса AdMob}}


%% ----- SKILLS SECTION -----
\newcommand{\skillsWord}
{\langen{Programming Skills}\langru{Навыки}}
\newcommand{\skillProgLanguages}
{\langen{Languages}\langru{Языки прогр.}}
\newcommand{\skillMachineLearning}
{\langen{Machine Learning}\langru{Машинное обучение}}
\newcommand{\skillAutoTesting}
{\langen{Automation Testing}\langru{Автоматизация тестирования}}
\newcommand{\skillTools}
{\langen{Tools}\langru{Инструменты}}